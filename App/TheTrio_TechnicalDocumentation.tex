% Options for packages loaded elsewhere
\PassOptionsToPackage{unicode}{hyperref}
\PassOptionsToPackage{hyphens}{url}
%
\documentclass[
]{article}
\usepackage{amsmath,amssymb}
\usepackage{iftex}
\ifPDFTeX
  \usepackage[T1]{fontenc}
  \usepackage[utf8]{inputenc}
  \usepackage{textcomp} % provide euro and other symbols
\else % if luatex or xetex
  \usepackage{unicode-math} % this also loads fontspec
  \defaultfontfeatures{Scale=MatchLowercase}
  \defaultfontfeatures[\rmfamily]{Ligatures=TeX,Scale=1}
\fi
\usepackage{lmodern}
\ifPDFTeX\else
  % xetex/luatex font selection
\fi
% Use upquote if available, for straight quotes in verbatim environments
\IfFileExists{upquote.sty}{\usepackage{upquote}}{}
\IfFileExists{microtype.sty}{% use microtype if available
  \usepackage[]{microtype}
  \UseMicrotypeSet[protrusion]{basicmath} % disable protrusion for tt fonts
}{}
\makeatletter
\@ifundefined{KOMAClassName}{% if non-KOMA class
  \IfFileExists{parskip.sty}{%
    \usepackage{parskip}
  }{% else
    \setlength{\parindent}{0pt}
    \setlength{\parskip}{6pt plus 2pt minus 1pt}}
}{% if KOMA class
  \KOMAoptions{parskip=half}}
\makeatother
\usepackage{xcolor}
\usepackage[margin=1in]{geometry}
\usepackage{color}
\usepackage{fancyvrb}
\newcommand{\VerbBar}{|}
\newcommand{\VERB}{\Verb[commandchars=\\\{\}]}
\DefineVerbatimEnvironment{Highlighting}{Verbatim}{commandchars=\\\{\}}
% Add ',fontsize=\small' for more characters per line
\usepackage{framed}
\definecolor{shadecolor}{RGB}{248,248,248}
\newenvironment{Shaded}{\begin{snugshade}}{\end{snugshade}}
\newcommand{\AlertTok}[1]{\textcolor[rgb]{0.94,0.16,0.16}{#1}}
\newcommand{\AnnotationTok}[1]{\textcolor[rgb]{0.56,0.35,0.01}{\textbf{\textit{#1}}}}
\newcommand{\AttributeTok}[1]{\textcolor[rgb]{0.13,0.29,0.53}{#1}}
\newcommand{\BaseNTok}[1]{\textcolor[rgb]{0.00,0.00,0.81}{#1}}
\newcommand{\BuiltInTok}[1]{#1}
\newcommand{\CharTok}[1]{\textcolor[rgb]{0.31,0.60,0.02}{#1}}
\newcommand{\CommentTok}[1]{\textcolor[rgb]{0.56,0.35,0.01}{\textit{#1}}}
\newcommand{\CommentVarTok}[1]{\textcolor[rgb]{0.56,0.35,0.01}{\textbf{\textit{#1}}}}
\newcommand{\ConstantTok}[1]{\textcolor[rgb]{0.56,0.35,0.01}{#1}}
\newcommand{\ControlFlowTok}[1]{\textcolor[rgb]{0.13,0.29,0.53}{\textbf{#1}}}
\newcommand{\DataTypeTok}[1]{\textcolor[rgb]{0.13,0.29,0.53}{#1}}
\newcommand{\DecValTok}[1]{\textcolor[rgb]{0.00,0.00,0.81}{#1}}
\newcommand{\DocumentationTok}[1]{\textcolor[rgb]{0.56,0.35,0.01}{\textbf{\textit{#1}}}}
\newcommand{\ErrorTok}[1]{\textcolor[rgb]{0.64,0.00,0.00}{\textbf{#1}}}
\newcommand{\ExtensionTok}[1]{#1}
\newcommand{\FloatTok}[1]{\textcolor[rgb]{0.00,0.00,0.81}{#1}}
\newcommand{\FunctionTok}[1]{\textcolor[rgb]{0.13,0.29,0.53}{\textbf{#1}}}
\newcommand{\ImportTok}[1]{#1}
\newcommand{\InformationTok}[1]{\textcolor[rgb]{0.56,0.35,0.01}{\textbf{\textit{#1}}}}
\newcommand{\KeywordTok}[1]{\textcolor[rgb]{0.13,0.29,0.53}{\textbf{#1}}}
\newcommand{\NormalTok}[1]{#1}
\newcommand{\OperatorTok}[1]{\textcolor[rgb]{0.81,0.36,0.00}{\textbf{#1}}}
\newcommand{\OtherTok}[1]{\textcolor[rgb]{0.56,0.35,0.01}{#1}}
\newcommand{\PreprocessorTok}[1]{\textcolor[rgb]{0.56,0.35,0.01}{\textit{#1}}}
\newcommand{\RegionMarkerTok}[1]{#1}
\newcommand{\SpecialCharTok}[1]{\textcolor[rgb]{0.81,0.36,0.00}{\textbf{#1}}}
\newcommand{\SpecialStringTok}[1]{\textcolor[rgb]{0.31,0.60,0.02}{#1}}
\newcommand{\StringTok}[1]{\textcolor[rgb]{0.31,0.60,0.02}{#1}}
\newcommand{\VariableTok}[1]{\textcolor[rgb]{0.00,0.00,0.00}{#1}}
\newcommand{\VerbatimStringTok}[1]{\textcolor[rgb]{0.31,0.60,0.02}{#1}}
\newcommand{\WarningTok}[1]{\textcolor[rgb]{0.56,0.35,0.01}{\textbf{\textit{#1}}}}
\usepackage{graphicx}
\makeatletter
\def\maxwidth{\ifdim\Gin@nat@width>\linewidth\linewidth\else\Gin@nat@width\fi}
\def\maxheight{\ifdim\Gin@nat@height>\textheight\textheight\else\Gin@nat@height\fi}
\makeatother
% Scale images if necessary, so that they will not overflow the page
% margins by default, and it is still possible to overwrite the defaults
% using explicit options in \includegraphics[width, height, ...]{}
\setkeys{Gin}{width=\maxwidth,height=\maxheight,keepaspectratio}
% Set default figure placement to htbp
\makeatletter
\def\fps@figure{htbp}
\makeatother
\setlength{\emergencystretch}{3em} % prevent overfull lines
\providecommand{\tightlist}{%
  \setlength{\itemsep}{0pt}\setlength{\parskip}{0pt}}
\setcounter{secnumdepth}{-\maxdimen} % remove section numbering
\ifLuaTeX
  \usepackage{selnolig}  % disable illegal ligatures
\fi
\IfFileExists{bookmark.sty}{\usepackage{bookmark}}{\usepackage{hyperref}}
\IfFileExists{xurl.sty}{\usepackage{xurl}}{} % add URL line breaks if available
\urlstyle{same}
\hypersetup{
  pdftitle={Technical Documentation},
  pdfauthor={The Trio},
  hidelinks,
  pdfcreator={LaTeX via pandoc}}

\title{Technical Documentation}
\author{The Trio}
\date{2025-04-18}

\begin{document}
\maketitle

\hypertarget{i.-introduction}{%
\section{I. Introduction}\label{i.-introduction}}

\hypertarget{project-overview}{%
\subsection{1. Project overview}\label{project-overview}}

Macroeconomic forecasting plays a critical role in shaping national
policy, but it faces a major challenge where real-time GDP data is often
revised over time, making forecasting decisions difficult. Research
(Croushore and Stark, 2001) has shown that forecasts using real-time
data tend to be less accurate than those using revised data whereby
traditional models may not reflect the actual economic situation.

Our project addresses this issue by developing a web-based interactive
application that allows users to benchmark various time-series
forecasting models; using real-time and revised GDP data. Our
application enables users to visualize and compare the accuracy and
robustness of different models across data vintages and forecast
horizons.

Our main goal is to help policymakers and economists evaluate how
forecasts would have performed in real-time, quantify the impact of data
revisions, and identify which models remain reliable under changing
economic conditions.

\hypertarget{overall-design}{%
\subsection{2. Overall design}\label{overall-design}}

Our application consist of two main tabs: Dataset and Model. The Dataset
Tab is where we allow user to either work with our provided sample
dataset or upload their own. Here user will also set up their interest
forecasting period. We also provide a preview of the cleaned data from
the starting period, along with visualizations to help users to examine
the differences in current and vintage data.

The Model tab allows users to select forecasting models for evaluation
and customize them by specifying key parameters and features. This
interactive setup ensures that users can experiment with different
models and directly observe their impact on forecasting performance.
Error metrics and visualization will be provided to assess the results.

\hypertarget{data-and-methodology}{%
\subsubsection{3. Data and methodology}\label{data-and-methodology}}

Our primary data source is available at
\url{https://www.philadelphiafed.org/surveys-and-data/real-time-data-research/routput}
We also use the FRED API to source additional economic data for model
enhancement.

In this project, we evaluate model performance using two different data
settings: vintage data (data available at the time of forecasting) and
latest vintage data (revised data). This comparison helps quantify the
impact of data revisions on forecast accuracy.

We focus on three models: - Autoregressive (AR): A benchmark model that
uses only past values of GDP growth to generate forecasts. -
Autoregressive Distributed Lag (ADL): Extends the AR model by
incorporating additional economic indicators. - K-Nearest Neighbors
(KNN): A non-parametric machine learning model that relies on historical
patterns

Model performance is assessed using two standard forecasting error
metrics: Root Mean Squared Error (RMSE) and Mean Absolute Error (MAE).
These metrics allow us to rank models consistently based on both
accuracy and robustness across different data vintages.

\hypertarget{ii.-application-backend}{%
\section{II. Application Backend}\label{ii.-application-backend}}

\hypertarget{application-framework}{%
\subsection{1. Application Framework}\label{application-framework}}

Before moving into building the application, it is necessary to go
through our logic. The framework of how our forecast works is as follow:
- Current Vintage Prediction: For each point we want to forecast, we
will use the vintage data at that point to train the model and generate
forecast. For instance, a forecast for 2000Q1 GDP growth will use the
2000Q1 vintage data to generate forecast. This is the real-time data
that we have in 2000Q1 about all other dates. - Latest Vintage
Prediction: We will use the latest vintage data for training our model
(2025Q1), however, we use only use data up until the time of prediction.
Thus, a 2000Q1 forecast with the latest data will only use GDP Growth up
until 1999Q4. - Evaluation: We will then generate two sequences of
forecasts, and calculate the error metrics of the sequences with the
real value in the latest vintage data. The reason is that this most
recent data is what closest to the truth by constantly going through
revision.

\hypertarget{data-preparation}{%
\subsection{2. Data Preparation}\label{data-preparation}}

For data selection, users can either work with our provided sample
dataset or upload their own. For sample dataset, we are using the
real-time dataset available here for download:
\url{https://www.philadelphiafed.org/surveys-and-data/real-time-data-research/routput}

For upload option, we currently support uploads in csv, xlsx, or json
formats. Additionally, the uploaded dataset must follow a structure
similar to the FRED data format to ensure compatibility with our
processing.

Our initial plan was to incorporate both quarter vintages and monthly
vintages. In order to do this, we need to detect the frequency of the
dataset that users chose. This is where we define a function that can
detect the frequency of dataset as below.

\begin{Shaded}
\begin{Highlighting}[]
\NormalTok{detect\_frequency }\OtherTok{\textless{}{-}} \ControlFlowTok{function}\NormalTok{(data) \{}
    \CommentTok{\# Get first vintage column name}
\NormalTok{    first\_col }\OtherTok{\textless{}{-}} \FunctionTok{names}\NormalTok{(data)[}\DecValTok{2}\NormalTok{]}
    
    \ControlFlowTok{if}\NormalTok{ (}\FunctionTok{str\_detect}\NormalTok{(first\_col, }\StringTok{"M}\SpecialCharTok{\textbackslash{}\textbackslash{}}\StringTok{d+$"}\NormalTok{)) \{}
      \FunctionTok{return}\NormalTok{(}\StringTok{"monthly"}\NormalTok{)}
\NormalTok{    \} }\ControlFlowTok{else} \ControlFlowTok{if}\NormalTok{ (}\FunctionTok{str\_detect}\NormalTok{(first\_col, }\StringTok{"Q}\SpecialCharTok{\textbackslash{}\textbackslash{}}\StringTok{d$"}\NormalTok{)) \{}
      \FunctionTok{return}\NormalTok{(}\StringTok{"quarterly"}\NormalTok{)}
\NormalTok{    \} }\ControlFlowTok{else}\NormalTok{ \{}
      \FunctionTok{stop}\NormalTok{(}\StringTok{"Unsupported Uploaded File!"}\NormalTok{)}
\NormalTok{    \}}
\NormalTok{  \}}
\end{Highlighting}
\end{Shaded}

After we have select our data, we start our data cleaning process. We
begin by cleaning the column names, which are in the format
``ROUTPUT65Q1'', to extract the correct year and quarter. For each
subsequent column, we compare the last two digits of the year with the
previous one: If the new year is greater than or equal to the previous,
we keep the current prefix.

If it's smaller, this signals a rollover into the next century, so we
increment the prefix by 1 (e.g., from ``19'' to ``20'').

For example, from ``98'' to ``99'', we stay in the 1900s. When it shifts
from ``99'' to ``00'', we move to the 2000s.

\begin{Shaded}
\begin{Highlighting}[]
\DocumentationTok{\#\# Get the right century for the year}
\NormalTok{  clean\_columns }\OtherTok{\textless{}{-}} \ControlFlowTok{function}\NormalTok{(data) \{}
    
\NormalTok{    data\_cols }\OtherTok{\textless{}{-}} \FunctionTok{names}\NormalTok{(data)[}\SpecialCharTok{{-}}\DecValTok{1}\NormalTok{] }\SpecialCharTok{\%\textgreater{}\%}
      \FunctionTok{str\_remove}\NormalTok{(}\AttributeTok{pattern =} \StringTok{"ROUTPUT"}\NormalTok{)}
    
\NormalTok{    yy }\OtherTok{\textless{}{-}} \FunctionTok{str\_sub}\NormalTok{(data\_cols, }\AttributeTok{start =} \DecValTok{1}\NormalTok{, }\AttributeTok{end =} \DecValTok{2}\NormalTok{)}
\NormalTok{    prev\_yy }\OtherTok{\textless{}{-}}\NormalTok{ yy[}\DecValTok{1}\NormalTok{]}
\NormalTok{    century }\OtherTok{=} \DecValTok{19}
\NormalTok{    complete\_year }\OtherTok{\textless{}{-}} \FunctionTok{c}\NormalTok{()}
    \ControlFlowTok{for}\NormalTok{ (i }\ControlFlowTok{in} \DecValTok{1}\SpecialCharTok{:}\FunctionTok{length}\NormalTok{(yy)) \{}
\NormalTok{      cur\_yy }\OtherTok{\textless{}{-}}\NormalTok{ yy[i]}
      \ControlFlowTok{if}\NormalTok{ (}\FunctionTok{as.numeric}\NormalTok{(cur\_yy) }\SpecialCharTok{\textless{}} \FunctionTok{as.numeric}\NormalTok{(prev\_yy)) \{}
\NormalTok{        century }\OtherTok{=}\NormalTok{ century }\SpecialCharTok{+} \DecValTok{1}
\NormalTok{      \}}
\NormalTok{      complete\_year[i] }\OtherTok{\textless{}{-}} \FunctionTok{paste0}\NormalTok{(century, cur\_yy)}
\NormalTok{      prev\_yy }\OtherTok{\textless{}{-}}\NormalTok{ cur\_yy}
\NormalTok{    \}}
    \FunctionTok{return}\NormalTok{(complete\_year)}
\NormalTok{  \}}
\end{Highlighting}
\end{Shaded}

This function will get us the right prefix for each column. As a result,
we can use this function and concatenate the prefix with the year and
quarter/month and get a better column name ,for instance,
``ROUTPUT65Q1'' to ``1965Q1''.

Next, we clean the dataset by reshaping into long format, allowing us to
extract and organize the columns into year, quarter, v\_year (vintage
year), v\_quarter/v\_month (vintage quarter/month) and the corresponding
GDP value. This structure makes it easier to filter for the
corresponding vintage in future analysis.

\begin{Shaded}
\begin{Highlighting}[]
\NormalTok{clean.data }\OtherTok{\textless{}{-}} \ControlFlowTok{function}\NormalTok{(data, }\AttributeTok{vintage\_freq =} \StringTok{"quarterly"}\NormalTok{) \{}
    
\NormalTok{    q }\OtherTok{\textless{}{-}} \FunctionTok{names}\NormalTok{(data)[}\SpecialCharTok{{-}}\DecValTok{1}\NormalTok{] }\SpecialCharTok{\%\textgreater{}\%} \FunctionTok{str\_remove}\NormalTok{(}\AttributeTok{pattern =} \StringTok{"ROUTPUT"}\NormalTok{) }\SpecialCharTok{\%\textgreater{}\%} \FunctionTok{str\_sub}\NormalTok{(}\AttributeTok{start =} \DecValTok{1}\NormalTok{)}
\NormalTok{    clean\_cols }\OtherTok{\textless{}{-}} \FunctionTok{paste0}\NormalTok{(}\FunctionTok{clean\_columns}\NormalTok{(data), q)}
    \FunctionTok{names}\NormalTok{(data)[}\SpecialCharTok{{-}}\DecValTok{1}\NormalTok{] }\OtherTok{\textless{}{-}}\NormalTok{ clean\_cols}
\NormalTok{    total\_col }\OtherTok{\textless{}{-}} \FunctionTok{length}\NormalTok{(}\FunctionTok{names}\NormalTok{(data))}
    \DocumentationTok{\#\# Clean the data}
\NormalTok{    clean\_data }\OtherTok{\textless{}{-}}\NormalTok{ data }\SpecialCharTok{\%\textgreater{}\%}
      \FunctionTok{mutate}\NormalTok{(}\FunctionTok{across}\NormalTok{(}\DecValTok{2}\SpecialCharTok{:}\NormalTok{total\_col, as.numeric)) }\SpecialCharTok{\%\textgreater{}\%}
      \FunctionTok{pivot\_longer}\NormalTok{(}\AttributeTok{cols =} \SpecialCharTok{{-}}\DecValTok{1}\NormalTok{, }\AttributeTok{names\_to =} \StringTok{"vintage"}\NormalTok{, }\AttributeTok{values\_to =} \StringTok{"current\_vintage"}\NormalTok{) }\SpecialCharTok{\%\textgreater{}\%}
      \FunctionTok{mutate}\NormalTok{(}\AttributeTok{year =} \FunctionTok{str\_sub}\NormalTok{(DATE, }\DecValTok{1}\NormalTok{,}\DecValTok{4}\NormalTok{),}
             \AttributeTok{quarter =} \FunctionTok{str\_sub}\NormalTok{(DATE, }\DecValTok{7}\NormalTok{,}\DecValTok{7}\NormalTok{), }
             \AttributeTok{v\_year =} \FunctionTok{str\_sub}\NormalTok{(vintage, }\DecValTok{1}\NormalTok{,}\DecValTok{4}\NormalTok{),}
             \AttributeTok{log\_current\_vintage =} \FunctionTok{log}\NormalTok{(current\_vintage)) }\SpecialCharTok{\%\textgreater{}\%}
      \FunctionTok{drop\_na}\NormalTok{()}
    
    \ControlFlowTok{if}\NormalTok{ (vintage\_freq }\SpecialCharTok{==} \StringTok{"quarterly"}\NormalTok{) \{}
\NormalTok{      final\_data }\OtherTok{\textless{}{-}}\NormalTok{ clean\_data }\SpecialCharTok{\%\textgreater{}\%} 
        \FunctionTok{mutate}\NormalTok{(}\AttributeTok{v\_quarter =} \FunctionTok{str\_extract}\NormalTok{(vintage, }\AttributeTok{pattern =} \StringTok{"(?\textless{}=Q).*$"}\NormalTok{)) }\SpecialCharTok{\%\textgreater{}\%}
        \FunctionTok{select}\NormalTok{(year, quarter, v\_year, v\_quarter, current\_vintage, log\_current\_vintage) }\SpecialCharTok{\%\textgreater{}\%}
        \FunctionTok{mutate}\NormalTok{(}\FunctionTok{across}\NormalTok{(}\DecValTok{1}\SpecialCharTok{:}\DecValTok{6}\NormalTok{, as.numeric))}
\NormalTok{    \}}
    \ControlFlowTok{else}\NormalTok{ \{}
\NormalTok{      final\_data }\OtherTok{\textless{}{-}}\NormalTok{ clean\_data }\SpecialCharTok{\%\textgreater{}\%}
        \FunctionTok{mutate}\NormalTok{(}\AttributeTok{v\_month =} \FunctionTok{str\_extract}\NormalTok{(vintage, }\AttributeTok{pattern =} \StringTok{"(?\textless{}=M).*$"}\NormalTok{)) }\SpecialCharTok{\%\textgreater{}\%}
        \FunctionTok{select}\NormalTok{(year, quarter, v\_year, v\_month, current\_vintage, log\_current\_vintage) }\SpecialCharTok{\%\textgreater{}\%}
        \FunctionTok{mutate}\NormalTok{(}\FunctionTok{across}\NormalTok{(}\DecValTok{1}\SpecialCharTok{:}\DecValTok{6}\NormalTok{, as.numeric))}
\NormalTok{    \}}
    
    
    \FunctionTok{return}\NormalTok{(final\_data)}
\NormalTok{  \}}
\end{Highlighting}
\end{Shaded}

As the project progressed, we chose to focus exclusively on quarterly
vintage data. This decision simplifies the modeling process while still
aligning with our primary goal: quantifying the impact of data
revisions. With that being said, it is completely possible to extend our
model to accompany monthly vintage data as the structure would be very
similar to how we handle quarterly data.

Moving on, we define a filtering function to extract data for a specific
vintage, based on the selected vintage year and quarter. This function
will take in the vintage year and vintage quarter and output the filter
table with the same structure, with the additional columns that we need
such as current vintage gdp level (current\_vintage) and vintage growth
(current\_growth)

\begin{Shaded}
\begin{Highlighting}[]
\NormalTok{filter\_function }\OtherTok{\textless{}{-}} \ControlFlowTok{function}\NormalTok{(v\_year1, v\_quarter1) \{}
    \FunctionTok{cleaned\_data}\NormalTok{() }\SpecialCharTok{\%\textgreater{}\%} \FunctionTok{filter}\NormalTok{(v\_year }\SpecialCharTok{==}\NormalTok{ v\_year1,}
\NormalTok{                              v\_quarter }\SpecialCharTok{==}\NormalTok{ v\_quarter1) }\SpecialCharTok{\%\textgreater{}\%}
      
      \FunctionTok{mutate}\NormalTok{(}\AttributeTok{lag\_current\_vintage =} \FunctionTok{lag}\NormalTok{(current\_vintage,}\DecValTok{1}\NormalTok{),}
             \AttributeTok{log\_lag\_current\_vintage =} \FunctionTok{log}\NormalTok{(lag\_current\_vintage),}
             \AttributeTok{current\_growth =} \DecValTok{400}\SpecialCharTok{*}\NormalTok{(log\_current\_vintage }\SpecialCharTok{{-}}\NormalTok{ log\_lag\_current\_vintage))}
    
\NormalTok{  \}}
\end{Highlighting}
\end{Shaded}

Due to missing data from earlier years in some vintages, we begin all
analyses from 1965, which is also the first available vintage in our
dataset. However, this introduces a challenge: if a user selects a
forecast starting point close to 1965, the model will have limited
historical data to train on. This lack of training data may negatively
impact model performance and forecast reliability. One solution would be
to restrict the range of vintages that users can choose to allow a
certain level of training size. This would be a potential area for
further research to determine the suitable range.

\hypertarget{model-construction}{%
\subsection{Model Construction}\label{model-construction}}

\hypertarget{autoregressive-ar}{%
\subsubsection{1. Autoregressive (AR)}\label{autoregressive-ar}}

We choose our baseline model to be the AR model. The AR model serves
well as the baseline model because it is a relatively intuitive model.
The idea behind the choice is that past growths (lags) of GDP could be
used to predict GDP growth in the next period.

First, we define a function to fit an AR model with lag p.~Since we are
using its own lag for regression, the function will only need to take in
the data for the target variable Y, accompany with p for the number of
lags and h for forecast horizon.

\begin{Shaded}
\begin{Highlighting}[]
\NormalTok{fitARp}\OtherTok{=}\ControlFlowTok{function}\NormalTok{(Y,p,h)\{}
    
    \CommentTok{\#Inputs: Y{-} predicted variable,  p {-} AR order, h {-}forecast horizon}
\NormalTok{    aux}\OtherTok{=}\FunctionTok{embed}\NormalTok{(Y,p}\SpecialCharTok{+}\NormalTok{h) }\CommentTok{\#create p lags + forecast horizon shift (=h option)}
\NormalTok{    y}\OtherTok{=}\NormalTok{aux[,}\DecValTok{1}\NormalTok{] }\CommentTok{\#  Y variable aligned/adjusted for missing data due to lags}
\NormalTok{    X}\OtherTok{=}\FunctionTok{as.matrix}\NormalTok{(aux[,}\SpecialCharTok{{-}}\FunctionTok{c}\NormalTok{(}\DecValTok{1}\SpecialCharTok{:}\NormalTok{(}\FunctionTok{ncol}\NormalTok{(Y)}\SpecialCharTok{*}\NormalTok{h))]) }\CommentTok{\# lags of Y corresponding to forecast horizon }
    \ControlFlowTok{if}\NormalTok{(h}\SpecialCharTok{==}\DecValTok{1}\NormalTok{)\{ }
\NormalTok{      X.out}\OtherTok{=}\FunctionTok{tail}\NormalTok{(aux,}\DecValTok{1}\NormalTok{)[}\DecValTok{1}\SpecialCharTok{:}\FunctionTok{ncol}\NormalTok{(X)] }\CommentTok{\#retrieve last p observations if one{-}step forecast }
\NormalTok{    \}}\ControlFlowTok{else}\NormalTok{\{}
\NormalTok{      X.out}\OtherTok{=}\NormalTok{aux[,}\SpecialCharTok{{-}}\FunctionTok{c}\NormalTok{(}\DecValTok{1}\SpecialCharTok{:}\NormalTok{(}\FunctionTok{ncol}\NormalTok{(Y)}\SpecialCharTok{*}\NormalTok{(h}\DecValTok{{-}1}\NormalTok{)))] }\CommentTok{\#delete first (h{-}1) columns of aux,  }
\NormalTok{      X.out}\OtherTok{=}\FunctionTok{tail}\NormalTok{(X.out,}\DecValTok{1}\NormalTok{)[}\DecValTok{1}\SpecialCharTok{:}\FunctionTok{ncol}\NormalTok{(X)] }\CommentTok{\#last p observations to predict T+1 }
\NormalTok{    \}}
    
\NormalTok{    model}\OtherTok{=}\FunctionTok{lm}\NormalTok{(y}\SpecialCharTok{\textasciitilde{}}\NormalTok{X) }\CommentTok{\#estimate direct h{-}step AR(p) by OLS }
\NormalTok{    coef}\OtherTok{=}\FunctionTok{coef}\NormalTok{(model) }\CommentTok{\#extract coefficients}
    \CommentTok{\#make a forecast using the last few observations: a direct h{-}step forecast.}
\NormalTok{    pred}\OtherTok{=}\FunctionTok{c}\NormalTok{(}\DecValTok{1}\NormalTok{,X.out)}\SpecialCharTok{\%*\%}\NormalTok{coef }
    
    \FunctionTok{return}\NormalTok{(}\FunctionTok{list}\NormalTok{(}\StringTok{"pred"}\OtherTok{=}\NormalTok{pred)) }
\NormalTok{  \}}
\end{Highlighting}
\end{Shaded}

\hypertarget{autoregressive-distributed-lag-adl}{%
\subsubsection{2. Autoregressive Distributed Lag
(ADL)}\label{autoregressive-distributed-lag-adl}}

\hypertarget{k-nearest-neighbours-knn}{%
\subsubsection{3. K Nearest Neighbours
(KNN)}\label{k-nearest-neighbours-knn}}

Our third model is K-Nearest Neighbors (KNN). KNN regression has been
widely studied for time series forecasting, and research has
consistently shown it to perform well in capturing nonlinear patterns in
the data (Lora et al., 2007; Zhang et al., 2017) We choose KNN to
explore the self-explanatory power of GDP growth and compare this with
parameter tuning models. We utilized the tsfknn package for using KNN
regression for time-series forecast. More information on the package can
be found here:

To generate an h-step ahead forecast, we use data only up to time T−h.
This approach ensures that our model does not unintentionally peek into
the future. We construct lag-based features by using the first four lags
of the target variable as autoregressive inputs. Since we are
forecasting multiple steps ahead (rather than just the next time point),
we also need to specify a multi-step ahead strategy. In our case, we use
the Multiple Input Multiple Output (MIMO) method, which allows the model
to generate all future predictions in one step. This avoids the
compounding error problem commonly encountered in recursive forecasting.
MIMO is therefore more stable and suitable for economic forecasting
tasks like GDP growth, where forecast horizons often extend several
quarters ahead. Then we extract the latest forecast which will be at
time T for our purpose. The loop will run through all forecast point in
the forecast list, generate forecast and assemble the result for
evalulation.

\begin{Shaded}
\begin{Highlighting}[]
\NormalTok{recursive\_prediction\_knn }\OtherTok{\textless{}{-}} \ControlFlowTok{function}\NormalTok{(h, k, cf) \{}
\NormalTok{    results }\OtherTok{=} \FunctionTok{data.frame}\NormalTok{(}\AttributeTok{year =} \FunctionTok{numeric}\NormalTok{(}\DecValTok{0}\NormalTok{), }\AttributeTok{quarter =} \FunctionTok{numeric}\NormalTok{(}\DecValTok{0}\NormalTok{), }\AttributeTok{cur\_forecast =} \FunctionTok{numeric}\NormalTok{(}\DecValTok{0}\NormalTok{), }\AttributeTok{latest\_forecast =} \FunctionTok{numeric}\NormalTok{(}\DecValTok{0}\NormalTok{))}
    \ControlFlowTok{for}\NormalTok{ (i }\ControlFlowTok{in} \DecValTok{1}\SpecialCharTok{:}\FunctionTok{nrow}\NormalTok{(}\FunctionTok{forecast\_list}\NormalTok{())) \{}
\NormalTok{      item }\OtherTok{=} \FunctionTok{forecast\_list}\NormalTok{()[i,]}
\NormalTok{      year\_f }\OtherTok{=}\NormalTok{ item}\SpecialCharTok{$}\NormalTok{year}
\NormalTok{      quarter\_f }\OtherTok{=}\NormalTok{ item}\SpecialCharTok{$}\NormalTok{quarter}
\NormalTok{      data }\OtherTok{\textless{}{-}} \FunctionTok{filter\_function}\NormalTok{(}\AttributeTok{v\_year1 =}\NormalTok{ year\_f, }\AttributeTok{v\_quarter1 =}\NormalTok{ quarter\_f) }\SpecialCharTok{\%\textgreater{}\%} \FunctionTok{left\_join}\NormalTok{(}\FunctionTok{latest\_vintage\_data}\NormalTok{(), }\AttributeTok{by =} \FunctionTok{c}\NormalTok{(}\StringTok{"year"}\NormalTok{, }\StringTok{"quarter"}\NormalTok{)) }\SpecialCharTok{\%\textgreater{}\%}
        \FunctionTok{select}\NormalTok{(year, quarter, current\_growth, latest\_growth) }\SpecialCharTok{\%\textgreater{}\%} \FunctionTok{filter}\NormalTok{(year }\SpecialCharTok{\textgreater{}=} \DecValTok{1965}\NormalTok{)}
\NormalTok{      data }\OtherTok{\textless{}{-}}\NormalTok{ data[}\DecValTok{1}\SpecialCharTok{:}\NormalTok{(}\FunctionTok{nrow}\NormalTok{(data)}\SpecialCharTok{{-}}\NormalTok{h}\SpecialCharTok{+}\DecValTok{1}\NormalTok{),] }\CommentTok{\# Extract data up until T {-} h}
\NormalTok{      train\_data\_cur }\OtherTok{\textless{}{-}}\NormalTok{ data }\SpecialCharTok{\%\textgreater{}\%} \FunctionTok{select}\NormalTok{(}\SpecialCharTok{{-}}\NormalTok{latest\_growth)}
\NormalTok{      train\_data\_latest }\OtherTok{\textless{}{-}}\NormalTok{ data }\SpecialCharTok{\%\textgreater{}\%} \FunctionTok{select}\NormalTok{(}\SpecialCharTok{{-}}\NormalTok{current\_growth)}
\NormalTok{      ts\_train\_cur }\OtherTok{\textless{}{-}} \FunctionTok{ts\_transform}\NormalTok{(train\_data\_cur) }
\NormalTok{      ts\_train\_latest }\OtherTok{\textless{}{-}} \FunctionTok{ts\_transform\_latest}\NormalTok{(train\_data\_latest)}
\NormalTok{      model\_current }\OtherTok{\textless{}{-}} \FunctionTok{knn\_forecasting}\NormalTok{(ts\_train\_cur, }\AttributeTok{h =}\NormalTok{ h, }\AttributeTok{lags =} \DecValTok{1}\SpecialCharTok{:}\DecValTok{4}\NormalTok{, }\AttributeTok{k =}\NormalTok{ k, }\AttributeTok{msas =} \StringTok{"MIMO"}\NormalTok{, }\AttributeTok{cf =}\NormalTok{ cf)}
\NormalTok{      cur\_pred }\OtherTok{\textless{}{-}} \FunctionTok{tail}\NormalTok{(model\_current}\SpecialCharTok{$}\NormalTok{prediction, }\DecValTok{1}\NormalTok{)}
\NormalTok{      model\_latest }\OtherTok{\textless{}{-}}  \FunctionTok{knn\_forecasting}\NormalTok{(ts\_train\_latest, }\AttributeTok{h =}\NormalTok{ h, }\AttributeTok{lags =} \DecValTok{1}\SpecialCharTok{:}\DecValTok{4}\NormalTok{, }\AttributeTok{k =}\NormalTok{ k, }\AttributeTok{msas =} \StringTok{"MIMO"}\NormalTok{, }\AttributeTok{cf =}\NormalTok{ cf)}
\NormalTok{      latest\_pred }\OtherTok{\textless{}{-}} \FunctionTok{tail}\NormalTok{(model\_current}\SpecialCharTok{$}\NormalTok{prediction, }\DecValTok{1}\NormalTok{)}
\NormalTok{      results[i, }\StringTok{\textquotesingle{}year\textquotesingle{}}\NormalTok{] }\OtherTok{=}\NormalTok{ year\_f}
\NormalTok{      results[i, }\StringTok{\textquotesingle{}quarter\textquotesingle{}}\NormalTok{] }\OtherTok{=}\NormalTok{ quarter\_f}
\NormalTok{      results[i, }\StringTok{\textquotesingle{}cur\_forecast\textquotesingle{}}\NormalTok{] }\OtherTok{=}\NormalTok{ cur\_pred}
\NormalTok{      results[i, }\StringTok{\textquotesingle{}latest\_forecast\textquotesingle{}}\NormalTok{] }\OtherTok{=}\NormalTok{ latest\_pred      }
    
\NormalTok{    \}}
    \FunctionTok{return}\NormalTok{(results)}
\NormalTok{  \}}
\end{Highlighting}
\end{Shaded}

\hypertarget{iii.-application-frontend}{%
\section{III. Application Frontend}\label{iii.-application-frontend}}

\end{document}
